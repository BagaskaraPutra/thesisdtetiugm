\begin{center}
	\textbf{[SAMPLE]}
\end{center}

Puji syukur ke hadirat Allah SWT yang telah melimpahkan rahmat dan barokah-Nya sehingga penulis dapat menyelesaikan tesis dengan judul "...". Laporan tesis ini disusun untuk memenuhi salah satu syarat dalam memperoleh gelar Master of Engineering (M.Eng.) pada Program Studi S2 Teknik Elektro Fakultas Teknik Universitas Gadjah Mada Yogyakarta.


Dalam melakukan penelitian dan penyusunan laporan tesis ini penulis telah mendapatkan banyak dukungan dan bantuan dari berbagai pihak. Penulis mengucapkan terima kasih yang tak terhingga kepada:

\begin{enumerate}
	\item <nama pembimbing utama + gelar> selaku dosen pembimbing utama, dan <nama pembimbing pendamping + gelar>selaku dosen pembimbing pendamping,yang telah dengan penuh kesabaran dan ketulusan memberikan ilmu dan bimbingan terbaik kepada penulis.
	
	\item <nama kaprodi + gelar> selaku Ketua Departemen Teknik Elektro dan Teknologi Informasi dan <nama kaminat + gelar> selaku Ketua Program Studi S2 Teknik Elektro Fakultas Teknik Universitas Gadjah Mada yang memberikan izin kepada penulis untuk belajar.
	
	\item Para Dosen Program Studi S2 Teknik Elektro Fakultas Teknik Universitas Gadjah Mada yang telah memberikan bekal ilmu kepada penulis.
	
	\item Para Karyawan/wati Program Studi S2 Teknik Elektro Fakultas Teknik Universitas Gadjah Mada yang telah membantu penulis dalam proses belajar.
	
	\item ........dst

\end{enumerate}

Penulis menyadari sepenuhnya bahwa laporan tesis ini masih jauh dari sempurna, untuk itu semua jenis saran, kritik dan masukan yang bersifat membangun sangat penulis harapkan. Akhir kata, semoga tulisan ini dapat memberikan manfaat dan memberikan wawasan tambahan bagi para pembaca dan khususnya bagi penulis sendiri.

\begin{flushright}
	\begin{tabular}{c}
		Yogyakarta, 13 Maret 2017 \\
		\vspace{1cm} \\
		Canggih
	\end{tabular}
\end{flushright}